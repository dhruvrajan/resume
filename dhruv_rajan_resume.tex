%!TEX TS-program = xelatex
\documentclass[]{friggeri-cv}
\addbibresource{bibliography.bib}






\usepackage{fontspec}
\usepackage{color}
\newcommand{\longhorn}{\includegraphics[scale=0.008]{longhorn.png}}
\begin{document}
\header{dhruv}{rajan}
       {\\}

% In the aside, each new line forces a line break
\begin{aside}
  \section{contact}
%    27930
%    Roble Blanco Drive
%    Los Altos Hills, CA 94022
    ~
    +1 (650) 229-4572
    \href{mailto:dhruv@cs.utexas.edu}{dhruv@cs.utexas.edu}
    %\href{mailto:dhruv.rajan@utexas.edu}{dhruv.rajan@utexas.edu}
%    \href{mailto:dhruv@krishnaprem.com}{dhruv@krishnaprem.com}
    ~
    \href{https://www.linkedin.com/in/dhruvrajan}{in://dhruvrajan}
    %\href{https://dhruvrajan.github.io}{dhruvrajan.github.io}
    \href{https://github.com/dhruvrajan}{github://dhruvrajan}
    %\href{https://atmospherejs.com/dhruv/}{atmospherejs://dhruv}
 \section{interests}
    artificial intelligence, machine learning, randomization, statistical learning theory
  	%probabilistic programming, \\bayesian machine learning,  \\functional \\programming
	%artificial intelligence, natural language 
	%processing, \\ machine learning,
	%functional programming, computer vision, computational music, robotics
%  \section{hobbies}
 % 	 guitar, tennis, cross-country, woodworking
\end{aside}

%\section{objective}
%Software development, employing cutting-edge statistical and natural language technologies to provide
%compelling user value.

\section{technical skills}
	Languages: Python, Java, C/C++, Haskell, Unix, Git, HTML5, CSS3, JavaScript, SQL, {\fontfamily{cmr}\selectfont \LaTeX} \\
	Packages: ~~nltk3, numpy, pandas, scipy, scikit-learn, OpenCV
    %Python (nltk3, numpy, pandas, scipy, scikit-learn, OpenCV), Haskell, Java, C++, C, \\ HTML5, CSS3, 
    %JavaScript, SQL, Unix,  x86 assembly, Git,  {\fontfamily{cmr}\selectfont \LaTeX }


\section{work experience}

\begin{entrylist}
\entry
 {May 2018}
  {Uber Inc.}
  {Palo Alto, CA}
  {\emph{\thinfont Software Engineering Intern (Maps)} \\
  	}

\entry
 {2016}
  {Gnetic, Inc. (\href{https://www.gneticinc.com}{www.gnetic.com})}
  {Palo Alto, CA}
  {\emph{\thinfont Software Engineer} \\
  	Put together and programmed electrical system for a prototype using an Arduino and RaspberryPI,
	and an Android controller.}
  %------------------------------------------------
\entry
{2014}
{Declara, Inc. (\href{https://www.declara.com}{www.declara.com})}
{Palo Alto, CA}
{\emph{\thinfont Data Science Intern} \\
Developed text tagging system for online documents, using python and techniques from text-retrieval, machine learning, and natural language processing.}
%------------------------------------------------
\entry
{2014}
{Whodini, Inc. }%(\href{https://www.whodini.com}{www.whodini.com})}
{(acquired by Declara, Inc. in 2014)~~~~~~~~~~~Los Altos, CA}
{\emph{\thinfont Software Engineering Intern} \\
Worked on topic-classification pipelines for workforce engagement and expert identification.}
\end{entrylist}


\section{education}

\begin{entrylist}
  \schoolentry
  	{2016-2020}
	{The University of Texas at Austin~~~~~\longhorn}
	{Austin, TX}
	{B.S. Computer Science, Turing Scholar}
	{CS 331H: Algorithms \& Complexity, CS 378: Randomized Algorithms, CS 378H: Data Mining, \\CS 343H Artificial Intelligence, CS 353: Theory of Computation, CS 439H Operating Systems,\\ CS 429H: Computer Org \& Architecture, CS 314H Data Structures, CS 311H Discrete Math, \\ M 340L Linear Algebra, M 362K Probability 1, Multivariable \& Vector Calculus}
	
  \schoolentry
  	{2012-2016}
	{Gunn High School}
	{Palo Alto, CA}
	{High School Diploma}
	{
	Online: Intro to Computational Thinking and Data Science (MIT 6.00x), \\
	Algorithms 1, Part 1 (Princeton), Principles of Functional Programming in Haskell (TU Delft)}
 \end{entrylist}

%\textbf{Additional Coursework/Exams} \\ \\ \\
%\begin{tabular}{l | c}
%	Course/Exam\, &\, Grade \\ \hline
%	MIT 600.x Introduction to Computational Thinking and Data Science\, & A \\
%	Introductory Physics 1 with Laboratory (Georgia Tech) & A \\
%	Introduction to Java (SJSU) & A \\
%	Introduction to Inferential Statistics (SJSU) & A \\
%	AP Calculus BC (self-study)& 5/5 \\
%	AP Physics C: Mechanics (self-study)& 5/5 \\
%	AP Physics C: Electricity \& Magnetism (self-study)& 5/5 \\
%	AP Statistics (self-study)& 4/5 \\
%	AP Chemistry & 5/5
%\end{tabular} \\

\section{projects}

\begin{entrylist}
\entry
  {2017}
  {UT Austin Computer Science}
  {Austin, TX}
  {\emph{\thinfont Undergraduate Researcher} \\
  	Worked on a system for functional probabilistic programming in Haskell.}
  \entry{2016}
  {scrivener.ai}
  {MHacks; Detroit, MI}
  {Using NLP and Speech Recognition to structurally transcribe audio from meetings to increase productivity.  \href{https://github.com/dhruvrajan/scrivener.ai}{https://github.com/dhruvrajan/scrivener.ai}}
  \entry{2013--2016}
{GRTPyFramework (GRT \# 192, \href{https://www.gunnrobotics.com}{gunnrobotics.com})}
{Palo Alto, CA}
{Developed robotics control system in python for Gunn's entry in the FIRST robotics competition. Won the \textit{Innovation in Control} award at the 2016 Milwaukee Regional for automatic vision-aided aiming, and controllable design. See (\href{https://github.com/grt192}{github://grt192}).}
\entry{2014}
  {15th Place in picoCTF}
  {Carnegie Mellon picoCTF}
  {Competed in a cyber-security ``capture-the-flag'' tournament; won 15th place out of ~3000 teams.}


\end{entrylist}

 
%\section{awards}
%\begin{entrylist}
%  %\entry{2014, 2016}
%  %{AP Scholar with Distinction, National AP Scholar}
%  %{CollegeBoard}{}
%%  \entry{2016}
%  {Andrea Erzberger Physics Scholarship}
%  {Ms. Andrea Erzberger, Gunn High School}
%  {\href{http://www.paloaltoonline.com/obituaries/memorials/andria-erzberger}
%  {http://www.paloaltoonline.com/obituaries/memorials/andria-erzberger}}
%  \entry{2016}
%  {Innovation In Control}{Milwaukee FRC Regional, Rockwell Automation}{Celebrates an innovative control system or application of control
%  components---electrical, mechanical, or software---to provide unique machine functions.}
  




\end{document}
